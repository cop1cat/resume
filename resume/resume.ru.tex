\documentclass{resume}
\usepackage[russian]{babel}
\usepackage[T2A]{fontenc}
\usepackage[utf8]{inputenc}
\usepackage[left=0.4in,top=0.4in,right=0.4in,bottom=0.4in]{geometry} % Поля документа
\newcommand{\tab}[1]{\hspace{.2667\textwidth}\rlap{#1}} 
\newcommand{\itab}[1]{\hspace{0em}\rlap{#1}}
\name{Даниил Спиридонов}
\address{+7(936) 505-1245 \\ Москва, Россия} 
\address{\href{mailto:sprv-4@yandex.ru}{sprv-4@yandex.ru} \\ \href{https://github.com/notaskynet}{github.com/notaskynet}}

\begin{document}

%----------------------------------------------------------------------------------------
%	ОБО МНЕ
%----------------------------------------------------------------------------------------
\begin{rSection}{ОБО МНЕ}
DS/MLE с опытом проектирования и внедрения data pipeline, реализации ML-моделей в продакшн-средах и интеграции различных систем для сервинга моделей. Постоянно совершенствую технические навыки, участвую в хакатонах и личных проектах.
\end{rSection}

%----------------------------------------------------------------------------------------
%	ОБРАЗОВАНИЕ
%----------------------------------------------------------------------------------------
\begin{rSection}{ОБРАЗОВАНИЕ}
{\bf Специалитет, направление "Компьютерная безопасность" \\ Национальный исследовательский университет "Высшая школа экономики" (НИУ ВШЭ)}, \hfill {Ожидается в 2028}\\
\textit{Московский институт электроники и математики} 
\end{rSection}

%----------------------------------------------------------------------------------------
%	ТЕХНИЧЕСКИЕ НАВЫКИ	
%----------------------------------------------------------------------------------------
\begin{rSection}{ТЕХНИЧЕСКИЕ НАВЫКИ}
\begin{tabular}{ @{} >{\bfseries}l @{\hspace{6ex}} l }
Языки программирования: & Python, C++ \\
Машинное обучение: & CatBoost, XGBoost, Transformers, LangChain, PyTorch, Scikit-Learn \\
Аналитика данных: & Pandas, NumPy, SQL \\
ML Ops: & Docker, Git, Airflow, MLflow, CI/CD \\
\end{tabular}
\end{rSection}

%----------------------------------------------------------------------------------------
%	ОПЫТ РАБОТЫ
%----------------------------------------------------------------------------------------
\begin{rSection}{ОПЫТ РАБОТЫ}

\textbf{ML-инженер}  
\textit{РТ-Информационная безопасность} | Дек 2024 – настоящее время  
\begin{itemize}
    \itemsep -3pt {} 
    \item Разработка и внедрение ML-моделей для управления инцидентами информационной безопасности.
    \item Проектирование и реализация пайплайнов сбора и обработки данных.
    \item Создание систем проверки качества данных для обеспечения надежности ML-задач.
\end{itemize}

\end{rSection}

%----------------------------------------------------------------------------------------
%	ПРОЕКТЫ
%----------------------------------------------------------------------------------------
\begin{rSection}{ПРОЕКТЫ}

\textbf{Claw Engine: модуль искусственного интеллекта}  
\textit{НИУ ВШЭ}  
\begin{itemize}
    \itemsep -3pt {} 
    \item Участие в разработке библиотеки для обучения агентов в Unity и её интеграция в Claw Engine.
    \item Разработка summarizer'а отзывов в AI-модуле: анализ отзывов и формирование структурированного контента в виде пунктов, использование LangChain для обработки естественного языка.
\end{itemize}

\end{rSection}

\end{document}
