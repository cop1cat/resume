\documentclass[a4paper,11pt]{article}

% --- Encoding and language ---
\usepackage[T2A]{fontenc}
\usepackage[utf8]{inputenc}
\usepackage[russian]{babel}

% --- Page layout and formatting ---
\usepackage[left=2cm,right=2cm,top=2cm,bottom=2cm]{geometry}
\usepackage{titlesec}
\usepackage{enumitem}
\usepackage{hyperref}
\usepackage{xcolor}

% --- Link colors ---
\hypersetup{
  colorlinks=true,
  urlcolor=blue,
  linkcolor=black
}

% --- Section formatting ---
\titleformat{\section}
  {\large\bfseries}
  {}
  {0pt}
  {}
  [\titlerule]

\titlespacing{\section}{0pt}{1em}{0.5em}
\setlist[itemize]{leftmargin=1.2cm, itemsep=2pt}

\begin{document}
\pagestyle{empty}

% --- Header ---
\begin{center}
{\LARGE \textbf{ДАНИИЛ СПИРИДОНОВ}}\\[6pt]
Телефон: +7 (936) 505-12-45 • Telegram: @daspiridonov\\
\href{mailto:sprv-4@yandex.ru}{sprv-4@yandex.ru} • \href{https://www.linkedin.com/in/daniel-spiridonov/}{linkedin.com/in/daniel-spiridonov}
\end{center}

% --- About ---
\section*{О СЕБЕ}
Инженер машинного обучения, специализируюсь на классическом ML и NLP. Разрабатываю продакшн-системы и автоматизирую ML-процессы. Обучаюсь по направлению «Компьютерная безопасность» в НИУ ВШЭ. Интересуюсь полным циклом ML: моделирование, пайплайны и инфраструктура.

% --- Experience ---
\section*{ОПЫТ РАБОТЫ}
\textbf{РТ-ИБ --- Инженер машинного обучения}\hfill \textit{Декабрь 2024 --- настоящее время}

\begin{itemize}
  \item Разработал автоматизированную систему классификации ложноположительных оповещений в событиях информационной безопасности с фильтрацией на основе правил
  \item Разработал автоматизированную систему генерации правил для фильтрации инцидентов информационной безопасности с использованием LLM и регулярных выражений
  \item Провел рефакторинг автоматизированной системы корреляции инцидентов информационной безопасности: переработал монолитную архитектуру в модульную с независимыми компонентами
  \item Разработал систему мониторинга и обнаружения аномалий для событий и инцидентов информационной безопасности: выявление падений, краткосрочных всплесков и долгосрочных трендов
  \item Разработал внутренние библиотеки для построения пайплайнов обработки ML-данных и создания ансамблей из модулей с различными циклами обучения
  \item Разработал ETL-систему для автоматизированного сбора и нормализации данных из нестандартизированных JSON-источников
  \item Внедрил CI/CD для ML-сервисов: унифицированный формат развертывания и автоматизированный деплой приложений
  \item Внедрил стандарты качества кода: линтеры, статические анализаторы, docstring для единого стиля разработки в команде
  \item Стандартизировал разработку DAG в Airflow для ML-пайплайнов
  \item Внедрил MinIO для версионирования датасетов и артефактов моделей
\end{itemize}

% --- Skills ---
\section*{ТЕХНИЧЕСКИЕ НАВЫКИ}
\textbf{Программирование:} Python (основной), C++, SQL

\textbf{ML/DL:} scikit-learn, CatBoost, XGBoost, PyTorch, vLLM

\textbf{Обработка данных:} pandas, numpy

\textbf{MLOps:} GitLab CI/CD, Linux, Docker, docker-compose

\textbf{Веб-фреймворки:} FastAPI

\textbf{Инфраструктура:} Airflow, MLflow, Langfuse, MinIO

\textbf{Визуализация:} Grafana, matplotlib, seaborn, plotly

\textbf{Инструменты разработки:} ruff, mypy, uv, git

% --- Education ---
\section*{ОБРАЗОВАНИЕ}
\textbf{НИУ ВШЭ}, Бакалавр компьютерных наук\\
Компьютерная и информационная безопасность\\
\textit{2022 --- 2028 (ожидается)}

\end{document}