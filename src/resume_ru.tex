\documentclass[a4paper,11pt]{article}

% --- XeLaTeX packages for Russian ---
\usepackage{fontspec}
\usepackage{polyglossia}
\setmainlanguage{russian}
\setotherlanguage{english}

% --- Fonts ---
% --- Fonts ---
\setmainfont{CMU Serif}[
  Extension=.otf,
  UprightFont=*-Roman,
  BoldFont=*-Bold,
  ItalicFont=*-Italic,
  BoldItalicFont=*-BoldItalic
]
\newfontfamily\cyrillicfont{CMU Serif}[Script=Cyrillic]
\setsansfont{CMU Sans Serif}
\setmonofont{CMU Typewriter Text}

% --- Page layout and formatting ---
\usepackage[left=2cm,right=2cm,top=2cm,bottom=2cm]{geometry}
\usepackage{titlesec}
\usepackage{enumitem}
\usepackage{hyperref}
\usepackage{xcolor}

% --- Link colors ---
\hypersetup{
  colorlinks=true,
  urlcolor=blue,
  linkcolor=black
}

% --- Section formatting ---
\titleformat{\section}
  {\large\bfseries\uppercase}
  {}
  {0pt}
  {}
  [\titlerule]

\titlespacing{\section}{0pt}{1em}{0.5em}
\setlist[itemize]{leftmargin=1.2cm, itemsep=2pt}

\begin{document}
\pagestyle{empty}

% --- Header ---
\begin{center}
{\LARGE \textbf{ДАНИИЛ СПИРИДОНОВ}}\\[6pt]
Телефон: +7 (936) 505-12-45 • Telegram: @daspiridonov\\
\href{mailto:sprv-4@yandex.ru}{sprv-4@yandex.ru}
\end{center}

% --- Education ---
\section*{Образование}
\textbf{НИУ ВШЭ}, ОП «Компьютерная безопасность»\\
2022 --- по наст. (ожидается окончание в 2028)

% --- Experience ---
\section*{Опыт работы}
\textbf{РТ-ИБ --- ML-инженер}\hfill \textit{Декабрь 2024 --- настоящее время}

\begin{itemize}
  \item Разработал и внедрил модель для выявления ложноположительных инцидентов информационной безопасности.
  \item Разработал сервис для автоматического сбора и дообогащения данных, обеспечивающих работу модели в продакшне и проведение экспериментов.
  \item Работал с LLM-моделями для автоматизации процессов SOC-центра.
  \item Внедрил линтеры и статический анализ кода как стандарт разработки.
  \item Разработал общий пайплайн CI/CD для вывода сервисов в прод.
\end{itemize}

% --- Skills ---
\section*{Технические навыки}
\textbf{Языки программирования:} Python

\textbf{Фреймворки и библиотеки:} FastAPI, vLLM, pandas, numpy, sklearn, catboost, pytorch

\textbf{Инфраструктура:} docker, docker-compose, airflow, mlflow, minio, grafana

\textbf{Визуализация:} matplotlib, seaborn, plotly

\textbf{CI/CD:} gitlab CI

\textbf{Инструменты:} ruff, uv

% --- About ---
\section*{О себе}
Специализируюсь на машинном обучении и обработке данных, в частности классическом ML и NLP. Опыт в проектировании и продакшн-внедрении ML-сервисов, а также автоматизации разработки и мониторинга. Учусь на 4-м курсе направления «Компьютерная безопасность».

\end{document}